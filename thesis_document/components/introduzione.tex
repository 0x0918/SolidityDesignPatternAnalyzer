\chapter*{Abstract}\label{abstract}
Blockchain technologies assume a key role in Web3, the next generation of the World Wide Web, whose goal is to create a decentralized and autonomous Internet in which users have greater control and ownership of their online data and activities. Within Ethereum blockchain, automated digital contracts, called \textit{"smart-contracts"}, are executed when certain conditions are met. Smart-contracts are written in code by a developer and as such are susceptible to vulnerabilities and security problems. A coding error or vulnerability can allow an attacker to compromise the contract and cause extensive and irreparable damage. Therefore, it is important that smart-contracts are carefully written and tested to ensure that they are free of vulnerabilities and function as intended.\\
\\
The goal of this thesis is to study, analyze, automating the process, and report which \textit{design patterns} are currently used in the development of Ethereum blockchain smart-contracts.
\vspace{20pt}
\begin{center}
\large$\star\star\star$
\end{center}
\vspace{20pt}
Le tecnologie blockchain assumono un ruolo chiave nel Web3, la prossima generazione del World Wide Web, il cui obiettivo è creare un Internet decentralizzato e autonomo in cui gli utenti hanno maggiore controllo e proprietà dei propri dati e attività online. All'interno della blockchain Ethereum vengono eseguiti, quando determinate condizioni sono soddisfatte, dei contratti digitali automatizzati, denominati \textit{"smart-contract"}. Gli smart-contract sono scritti in codice da un programmatore e come tali sono suscettibili a vulnerabilità e problemi di sicurezza. Un errore di codifica o una vulnerabilità può permettere a un attaccante di compromettere il contratto e causare danni ingenti e irreparabili. Pertanto, è importante che gli smart-contract vengano scritti e testati con cura per garantire che siano privi di vulnerabilità e funzionino come previsto.\\
\\
L'obiettivo di questa tesi è studiare, analizzare, automatizzando il processo, e documentare quali \textit{design pattern} siano attualmente utilizzati nello sviluppo degli smart-contract della blockchain Ethereum.
 
