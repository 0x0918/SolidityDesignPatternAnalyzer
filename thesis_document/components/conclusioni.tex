\chapter{Conclusioni}
Termina così lo sviluppo dell'applicativo software proposto nella tesi.\\
\newline
\textit{Solidity Design Pattern Analyzer} è un software \textit{open-source} rilasciato con licenza \textit{MIT}, il cui codice sorgente è liberalmente accessibile nel relativo repository su GitHub\cite{github-repo}.\\
\newline
L'applicativo è capace di rilevare, nei limiti linguistici e delle dipendenze utilizzate, tutti e ventidue i design pattern documentati nella tesi, i cui relativi descriptor sono inclusi nel codice sorgente, ed è facilmente possibile, mediante la combinazione di controlli generici, definire nuovi descriptor per riconoscere design pattern futuri.\\
\newline
Si conclude la tesi elencando un'ipotetica serie di sviluppi futuri:
\begin{itemize}
	\item Introdurre il supporto di dizionari di lingue diverse dall'inglese al fine di permettere l'analisi degli smart-contract le cui informazioni utili, come nomi di variabili di stato o di funzioni, corrispondono a parole di lingue straniere;
	\item Implementare un sistema di raggruppamento e interdipendenza dei controlli allo scopo di rilevare costrutti più articolati;
	\item Progettare un'interfaccia grafica per un utilizzo più user-friendly dell'applicativo;
\end{itemize}
 
 