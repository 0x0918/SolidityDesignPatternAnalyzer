\chapter{Introduzione e Contesto}
L'attuale generazione della tecnologia World Wide Web, denominata \textit{Web2}, è caratterizzata da una maggiore interattività e partecipazione degli utenti rispetto alla precedente generazione, detta \textit{Web1}. In genere, con il termine Web2, si denota la transizione dalla semplice navigazione di siti web all'utilizzo, da parte degli utenti della rete, di servizi e applicazioni interattive, come ad esempio: i social network, le piattaforme di streaming e gli e-commerce.

Questa seconda generazione è anche caratterizzata da un \textit{problema di centralizzazione}, il quale consiste nel fatto che molte delle informazioni e dei dati che circolano in rete sono gestiti da un numero ristretto di grandi aziende. Ciò può portare a problemi di privacy, sicurezza e libertà d'espressione, in quanto queste aziende possono utilizzare i dati degli utenti per fini commerciali o per influenzare l'accesso a determinate informazioni. Inoltre, la centralizzazione introduce problemi di dipendenza e vulnerabilità, in quanto un'unica entità ha il controllo sui dati e sui servizi forniti, la quale, se attaccata attraverso vulnerabilità, può essere soggetta al furto di una grande mole di dati sensibili.

La prossima generazione, denominata \textit{Web3}, punta a risolvere questo problema attraverso la decentralizzazione e la creazione di applicazioni e servizi che non dipendono da un'unica entità o autorità centrale, ma che sono gestiti da una rete di nodi distribuiti. In questo modo, si mira a creare una rete più equa, sicura e resiliente, dove gli utenti hanno il controllo dei propri dati e delle proprie informazioni.

Di seguito in questo capitolo vengono presentati i concetti fondamentali del Web3, il tema d'interesse della tesi, le modalità e il lavoro svolto per lo svolgimento quest'ultimo e la struttura generale della tesi.
\section{Le blockchain}
Le blockchain sono una tecnologia di registro distribuito che consente la creazione di una rete peer-to-peer senza la necessità di intermediari. Ciò significa che le transazioni e le informazioni possono essere scambiate direttamente tra gli utenti senza la necessità di un'autorità centrale. Ciò rende le blockchain adatte per la creazione di un Internet decentralizzato in cui gli utenti hanno maggiore sicurezza, trasparenza e autonomia.\\
Gli smart contract sono una forma di contratto digitale automatizzato che si esegue automaticamente quando determinate condizioni sono soddisfatte. Essi sono scritti in un linguaggio di programmazione che consente la creazione di regole e logiche specifiche che vengono eseguite su una blockchain.

Gli smart contract consentono la creazione di accordi automatici tra le parti in modo da ridurre i tempi di elaborazione e i costi associati alle transazioni tradizionali. Inoltre, essi possono essere utilizzati per creare applicazioni decentralizzate (dApps) che utilizzano la tecnologia blockchain per garantire la sicurezza e la trasparenza delle transazioni.

Per quanto riguarda la sicurezza, gli smart contract sono scritti in codice e come tali sono suscettibili a vulnerabilità e problemi di sicurezza. Un errore di codifica o una vulnerabilità può permettere a un attaccante di compromettere il contratto e causare danni irreparabili. Pertanto, è importante che gli smart contract vengano scritti e testati con cura per garantire che siano privi di vulnerabilità e funzionino come previsto.

\section{Struttura Tesi}
Di seguito vengono illustrati gli argomenti trattati in ogni capitolo presente:
\begin{itemize}
	\item Nel \textit{capitolo 2} vengono introdotti gli strumenti utilizzati per lo svolgimento del tema della tesi;
	\item Nel \textit{capitolo 3} vengono descritti i design pattern individuati nello studio dell'attuale divulgazione scientifica;
	\item Nel \textit{capitolo 4} viene presentato il software di analisi statica automatica sviluppato;
	\item Nel \textit{capitolo 5} vengono riportati i risultati dell'analisi automatica di alcuni smart-contract open-source;
	\item Nell'\textit{appendice} vengono riportati i codici di riferimento dei design pattern usati per lo sviluppo del software;
\end{itemize}